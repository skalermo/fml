\documentclass[12pt,a4paper]{article}

\usepackage{polski}

% source code
\usepackage{listings}

% railroad diagrams
\usepackage{rail}

\usepackage{graphicx}

% correct footnotes placement
\usepackage[bottom]{footmisc}

% links
\usepackage{hyperref}
\hypersetup{
    colorlinks,
    citecolor=black,
    filecolor=black,
    linkcolor=black,
    urlcolor=black
}

% extra level of sections
\usepackage{titlesec}
\setcounter{secnumdepth}{4}
\setcounter{tocdepth}{4}
\titleformat{\paragraph}
{\normalfont\normalsize\bfseries}{\theparagraph}{1em}{}
\titlespacing*{\paragraph}
{0pt}{3.25ex plus 1ex minus .2ex}{1.5ex plus .2ex}

\usepackage{multirow} 
\usepackage{makecell}


\title{Dokumentacja do projektu z TKOM}
\author{Roman Moskalenko}
\date{}

\begin{document}

\maketitle
\tableofcontents
\pagebreak

\section{Treść zadania}
Interpretacja własnego języka z wbudowanym typem macierzy dwuwymiarowej.

\section{Projekt wstępny}
\subsection{Opis ogólny}

Fantastic Matrix Language (Fantastyczny język macierzowy, dalej FML) jest językiem skryptowym dynamicznie typowanym. Można go traktować jako uproszczone połączenie \emph{Pythona} i \emph{C}. \hyperlink{foo}{\hypertarget{foo_back}{link}}.

Wszystkie białe znaki (whitespace) są ignorowane
\footnote{Wyjątkiem są znaki będące zawartością stringa(zapisane pomiędzy nawiasów \textbf{"" }).}
 przez interpreter, dlatego nie zostaną uwzględnione na większości diagramów co pozwoli je uprościć. Także ignorowane są komentarze poprzedzone znakiem \textbf{\#} (jednoliniowe).

Interpreter musi być w stanie obsługiwać różne typy źródeł: string, plik, strumień.


%
% Słowa kluczowe
%
\subsection{Słowa kluczowe}

FML łącznie zawiera 14 słów kluczowych, które są przedstawione poniżej.

\begin{lstlisting}
and div do else for fun if 
in mod not or ret while
\end{lstlisting}

\begin{itemize}
\item Operatory logiczne: \textbf{and, not, or}

\item Konstrukcje pętlowe: \textbf{do, for, in, while}

\item Operatory arytmetyczne: \textbf{div, mod}

\item Konstrukcje warunkowe: \textbf{else, if}

\item Konstrukcje funkcji: \textbf{fun, ret}

\end{itemize}
\pagebreak

%
% Identyfikatory i typy danych
%
\subsection{Identyfikatory i typy danych}

\subsubsection{Indentyfikator}

Identyfikatorem jest zmienna, która może reprezentować jeden z trzech typów danych bądź funkcję.

FML oferuje 3 typy danych: \textbf{skalar}, \textbf{macierz} oraz \textbf{string}.

\subsubsection{Skalar}

Skalarem jest stała liczba. Możliwy jest zapis w notacji naukowej.

Przykłady:
\begin{itemize}
\item 0, 1, -12.3, 12.3e4, 12.3e-45.
\end{itemize}

\subsubsection{String}
String jest to łańcuch symboli poprzedzony i zakończony znakiem \textbf{"}. 
String nie może brać udziału w wyrażeniach arytmetycznych lub logicznych. Nie może być przechowywany w macierzach. Może być przekazywany jako argument w niektórych funkcjach, stosowany w przypisywaniach. Jest iterowalny.

\medskip
Przykłady:
\begin{itemize}

\item "Hello world!\verb+\n+"

\item "Ala nie ma\verb+\t+ kota"
\end{itemize}

\subsubsection{Macierz}
Macierz jest tablicą dwuwymiarową, która może zawierać tylko skalary.

Znak \textbf{,} jest używany dla separacji kolejnych elementów w jednym wierszu macierzy. Znak \textbf{;} jest używany dla separacji kolejnych wierszów. Wiersz macierzy nie może być pusty, chyba, że cała macierz jest pusta. Liczba elementów w każdym wierszu macierzy musi być taka sama.

\bigskip
Przykłady:

\begin{itemize}

\item {[ ]}

\item {[1, 2, 3, 4, 5]}

\item {[1, 2, 3; 4, 5, 6; 7, 8, 9]}

\item {[1, 2, 3, 4;\\
        5, 6, 7, 8  ]}

\end{itemize}

\subsubsection{Indeksowanie}

Odwołanie do elementów macierzy bądź stringa jest możliwe dzięki operatorowi \textbf{[ ]}. Do elementu macierzy można dostać się za pomocą dwóch indeksów odseparowanych przecinkiem: pierwszy specyfikuje wiersz macierzy, drugi -- kolumnę. Podając zamiast indeksu znak \textbf{:} można odwołać się do wszystkich elementów wiersza/kolumny. W przypadku ujemnego indeksu zostanie zwrócony element licząc od końca.

\medskip
Przykłady:

\begin{lstlisting}
m = [1, 2, 3;
     4, 5, 6]
     
m[0, 1] # 2

m[:, 1] # [2, 5]

m[1,:] # [4, 5, 6]
\end{lstlisting}

W przypadku podania tylko jednego indeksu bez separatora zostanie zwrócony element z wypłaszczonej macierzy. Indeksowanie stringa jest analogiczne do indeksowania w Pythonie.

\begin{lstlisting}
s = "Hello world!"

s[4] # o

m = [1, 2, 3, 4; 5, 6, 7, 8]

m[5] # 6
\end{lstlisting}

\subsubsection{Stała}

Stała (\emph{Constant}) jest pewną wartością używaną w wyrażeniach. Może to być skalar bądź macierz.

\pagebreak

\subsection{Wyrażenia i operatory}

\subsubsection{Operatory}
Operatory są przedstawione w tabeli poniżej.

\begin{table}[ht]
  \centering
  \begin{tabular}{ |c|c|c|c| }
    \hline
    \textbf{Priorytet} & \textbf{Operator} & \textbf{Opis} \\ [0.5ex] 
    \hline
    1 & ** & Potęgowanie \\
    \hline
    \multirow{2}{*}{2} & + - & Jednoargumentowy plus i minus \\
                       & not & Logiczne zaprzeczenie  \\
    \hline
    \multirow{2}{*}{3} & * / mod & Mnożenie, dzielenie oraz modulo \\
                       & div & Dzielenie całkowitoliczbowe \\
    \hline
    4 & + - & Dodawanie, odejmowanie \\ 
    \hline
    \multirow{2}{*}{5} & $<$ $<=$ & Operatory $<$ oraz $\leq$ \\
                       & $>$ $>=$ & Operatory $>$ oraz $\geq$ \\
    \hline
    6 & == != & Operatory $=$ oraz $\neq$ \\
    \hline
    7 & and or & Logiczny iloczyn i suma \\
    \hline
    8 & = & Przypisanie \\
    \hline
  \end{tabular}
\end{table}

\subsubsection{Wyrażenia}

Wyrażenie może być pojedynczą stałą, zmienną lub też pewną operacją na zmiennych. Kolejność wykonywanych operacji jest zdefiniowana za pomocą priorytetów operatorów opisanych powyżej. Także mogą być stosowane nawiasy okrągłe: \textbf{(} i \textbf{)}. Wyrażenie nie może być puste, nie może zawierać samych nawiasów.

\subsubsection{Przypisanie wartości identyfikatorowi}

Przypisanie jest szczególnym przypadkiem wyrażenia, jest dokonywane za pomocą operatora przypisania \textbf{=}. 

\subsection{Konstrukcje warunkowe, funkcje, pętle}

\subsubsection{Wyrażenia logiczne}

Zanim przejść do konstrukcji warunkowych trzeba zdefiniować jaką wartość może przyjmować wyrażenie logiczne. FML nie ma specjalnego typu wartości logicznych. Natomiast przyjęto takie założenia.

Wyrażenie logiczne jest \textbf{fałszywe} gdy jest równe jednej z poniższych wartości:

\begin{itemize}
  \item Skalar zerowy \textbf{0}.
  \item Pusta macierz \textbf{[ ]}.
\end{itemize}

W pozostałych przypadkach wyrażenie logiczne będzie \textbf{prawdziwe}.

Jeśli wyrażenie logiczne jest prawdziwe to jego wartość jest równa 1. W przypadku gdy wyrażenie logiczne nie jest prawdziwe wartość jego jest zerowa.

\subsubsection{Konstrukcje warunkowe}

Do konstrukcji warunkowych używane są słowa kluczowe \textbf{if} oraz \textbf{else}. Gramatyka języka umożliwia zagnieżdżanie tych konstrukcji.

Przykłady:
\lstset{language=Python}
\begin{lstlisting}
if (a<b) a=b;
else {
  a = a - 10;
  b = b * 2;
}

if (a)
  if (b) {
    b = c+d;
  }
  else
    a = c+d;
    
if (a) {
  if (b)
    b = b + 10;
} else { 
  a = a + 10;
}

\end{lstlisting}

\subsubsection{Funkcje}

Deklaracja funkcji rozpoczyna się od słowa kluczowego \textbf{fun}.
Po nim następuje identyfikator funkcji. Parametry funkcji należy opisać w nawiasach jako identyfikatory odseparowane przecinkiem. Nawiasy puste oznaczają brak parametrów. Na koniec jest ciało funkcji.

\medskip

Przykład:

\begin{lstlisting}
fun my_function(parameter1, parameter2)
  a = parameter1 + parameter2;
  
\end{lstlisting}

Aby zwrócić wartość przez funkcję należy użyć słowa kluczowego \textbf{ret}. Domyślną wartością zwracaną przez funkcje jest \textbf{0}.

\medskip
Przykład:

\begin{lstlisting}
fun my_function(parameter)
  ret parameter + 1;

\end{lstlisting}

Aby wywołać istniejącą funkcje należy podać jej identyfikator oraz argumenty wywołania w nawiasach okrągłych.

\medskip
Przykład:

\begin{lstlisting}
my_function();
my_function(1);
my_function(a, b);
\end{lstlisting}

Funkcje mogą być definiowane przez użytkownika tylko w globalnym zakresie programu (nie można definiować węwnatrz pętli, innych funkcji itd.)

\paragraph{Funkcje wbudowane}

\medskip
Do wbudowanych należa funkcje:

\begin{itemize}

  \item abs() -- zwraca wartość bezwzględną argumentu,
  \item len() -- zwraca długość (liczbę elementów) objektu,
  \item max() -- zwraca największą liczbę z podanej sekwencji,
  \item min() -- zwraca najmniejszą liczbę z podanej sekwencji,
  \item print() -- wypisuje na konsole podane argumenty,
  \item round() -- zwraca zaokrąglony skalar,
  \item shape() -- zwraca macierz zawierającą wymiary podanego argumentu,
  \item transp() -- odwraca podaną macierz.
  
\end{itemize}

\subsubsection{Pętle}

W FML wyróżnia się 3 typy pętli: \textbf{for loop}, \textbf{while loop}, \textbf{do while loop}.

\medskip

Pętle \textbf{for loop} przeznaczone do iterowania po macierzy lub stringu. Do tego używa się słowa kluczowego \textbf{in}, który rozwija iterowalny zbiór.

\medskip
Przykład wyliczenia sumy wszystkich elementów macierzy:

\begin{lstlisting}
m = [1, 2; 3.5, 4.5];
s = 0.0;
for (i in m) {
  s = s + i;
}
\end{lstlisting}

Pętla mogła również zostać zapisana w sposób następujący:
\begin{lstlisting}
...
for (i in m)
  s = s + i;
\end{lstlisting}

W pętlach \textbf{while} instrukcje są wykonywane póki jest spełniony warunek. W przypadku \textbf{do while loop} zawartość pętli uruchomi się co najmniej raz niezależnie od tego czy spełniony warunek pętli. Składnia jest analogiczna do języka \textbf{C}.

\medskip
Przykłady wypisania liczb od 1 do 10:

\begin{lstlisting}

i = 1
while (i <= 10) {
  print(i);
  i = i + 1;
}

i = 1;
do {
  print(i);
  i = i + 1;
} while (i <= 10);

\end{lstlisting}

FML nie rozpoznaje słów kluczowych \textbf{break} i \textbf{continue}. Nie jest możliwe ich użycie węwnątrz konstrukcji pętlowych.

\pagebreak

\section{Diagramy składniowe}

\begin{rail}
program : (statementList | functionDeclaration)+
\end{rail}

\begin{rail}
statementList : (statement)+ 
\end{rail}

\begin{rail}
compoundStatement : '<' statementList '>'
\end{rail} \footnote{Uwaga: zamiast \textbf{$< >$} na diagramie powinne stać znaki \textbf{$\{ \}$}. Latexowi z niewiadomego mi powodu nie podoba się wykorzystanie nawiasów klamrowych w diagramach.}

\begin{rail}
functionDeclaration : 'fun' identifier '(' (identifier + ',')?  ')' statement
\end{rail}

\begin{rail}
statement : (((('ret')? expression) | 'do while loop') ';') | 'while loop' | 'for loop' | 'if statement' | compoundStatement
\end{rail}

\begin{rail}
doWhileLoop : 'do' statement '(' logicalExpression ')'
\end{rail}

\begin{rail}
whileLoop : while '(' logicalExpression ')' statement
\end{rail}

\begin{rail}
forLoop : 'for' '(' identifier 'in' (identifier | matrix | string) ')' statement
\end{rail}

\begin{rail}
ifStatement : 'if' '(' logicalExpression ')' statement ('else' statement)?
\end{rail}

\begin{rail}
expression : logicalExpression | assignment
\end{rail}

\begin{rail}
assignment : identifier '=' (logicalExpression | string) 
\end{rail}

\begin{rail}
logicalExpression : logicalSubExpression (('or'  logicalSubExpression)*)
\end{rail}

\begin{rail}
logicalSubExpression : equalityExpression (('and'  equalityExpression)*)
\end{rail}

\begin{rail}
equalityExpression : comparisonExpression ((('==' | '!=') comparisonExpression)*)
\end{rail}

\begin{rail}
comparisonExpression : arithmeticExpression ((('<'|'<='|'>'|'>=') arithmeticExpression)*)
\end{rail}

\begin{rail}
arithmeticExpression : term ((('+' | '-') term)*)
\end{rail}

\begin{rail}
term : miniterm ((('*' | '/' | 'div' | 'mod') miniterm)*)
\end{rail}

\begin{rail}
miniterm : ('+' | '-' | not)? microterm
\end{rail}

\begin{rail}
microterm : factor ('**' factor)?
\end{rail}

\begin{rail}
factor : identifier | constant | functionCall | arraySubscripting |'(' expression ')'
\end{rail}

\begin{rail}
functionCall : identifier '(' (expression + ',')? ')'
\end{rail}

\begin{rail}
arraySubscripting : identifier '[' (expression | ((expression|':') ',' (expression|':'))) ']'
\end{rail}

\begin{rail}
identifier : letter ((letter ? | digit | 'underscore' ) +);     
\end{rail}

\begin{rail}
constant : scalar | matrix
\end{rail}

\begin{rail}
scalar : ''[scalarNoExp] (('e' | 'E') ('-' | '+')? (digit+))?;
scalarNoExp : integer ('.' (digit +))?;
integer : '-' ? ('0' | (digit[1-9] (digit +)));
\end{rail}

\begin{rail}
matrix : '[' ((row + ';')?) ']';
\end{rail}

\begin{rail}
row : expression + ',';
\end{rail}

\hspace*{-2cm}\vbox
{
  \begin{rail}
    string : '"' (('character' ('backslash' ('backslash' | '"' | 't' | 'n'))?)+)? '"';
  \end{rail}
}

gdzie \emph{character} jest dowolnym znakiem oprócz " i \verb+\+ oraz znaków kontrolnych.

\begin{rail}

  whitespace : (([space]'' | [linefeed]'' | [horizontal tab]'')+)?;  

\end{rail}

\end{document}