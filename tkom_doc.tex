\documentclass[12pt,a4paper]{article}


\usepackage{polski}
\usepackage{listings}
\usepackage{rail}
\usepackage{graphicx}
\usepackage{hyperref}
\hypersetup{
    colorlinks,
    citecolor=black,
    filecolor=black,
    linkcolor=black,
    urlcolor=black
}


\title{Dokumentacja do projektu z TKOM}

\author{Roman Moskalenko}
\date{}

\begin{document}
\maketitle
\tableofcontents
\pagebreak

\section{Treść zadania}
Interpreter własnego języka z wbudowanym typem macierzy dwuwymiarowej.

\section{Projekt wstępny}
\subsection{Założenia ogólne}

Fantastic Matrix Language (Fantastyczny język macierzowy, dalej FML) jest językiem skryptowym dynamicznie typowanym. Na ogół przypomina uproszczoną składnię Python'a oraz języka C, co jest pokazane na przykładach poniżej.

%
% Słowa kluczowe
%
\subsection{Słowa kluczowe}

FML łącznie zawiera 20 słów kluczowych, które są przedstawione poniżej.

\begin{lstlisting}
and break case continue default div do else for fun if 
in is mod not null or return switch while
\end{lstlisting}

\begin{itemize}
\item Operatory logiczne: \textbf{and, not, or}

\item Konstrukcja switch/case: \textbf{break, case, default, switch}

\item Konstrukcje pętlowe: \textbf{continue, do, for, in, while}

\item Operatory arytmetyczne: \textbf{div, mod}

\item Konstrukcje warunkowe: \textbf{else, if}

\item Konstrukcje funkcji: \textbf{fun, return}

\item Operator identyczności: \textbf{is}

\item Objekt pusty: \textbf{null}

\end{itemize}
\pagebreak

%
% Identyfikatory i typy danych
%
\subsection{Identyfikatory i typy danych}

\subsubsection{Indentyfikator}

Poniższy diagram ilustruje poprawne tworzenie identyfikatorów w FML.

%
% some railroad magic
%
\begin{rail}

identifier : letter ((letter ? | digit | 'underscore' ) +);
     
\end{rail}


FML oferuje 3 typy danych: \textbf{skalar}, \textbf{macierz} oraz \textbf{string}. 

\subsubsection{Skalar}

Skalarem jest liczba, zapis której opisany diagramem poniżej:

\begin{rail}

scalarNoExp : '-' ? ('0' | (digit[1-9] (digit +))) ('.' (digit +))? ;
scalar : ''[scalarNoExp] (('e' | 'E') ('-' | '+') (digit+))?

\end{rail}

Przykłady:
\begin{itemize}
\item 0, 1, -12.3, 12.3e4, 12.3e-45.
\end{itemize}

\pagebreak
\subsubsection{String}
String jest to łańcuch symboli. Diagram:

\hspace*{-2cm}\vbox
{
  \begin{rail}

    string : '"' (('character' ('backslash' ('backslash' | '"' | 't' | 'n'))?)+)? '"';

  \end{rail}
}

gdzie \emph{character} jest dowolnym znakiem oprócz " i \verb+\+ oraz znaków kontrolnych.

\medskip
Przykłady:
\begin{itemize}

\item "Hello world!\verb+\n+"

\item "Ala nie ma\verb+\t+ kota"
\end{itemize}

\subsubsection{Macierz}
Macierz jest tablicą dwuwymiarową, która może zawierać skalary, stringi, wartości \textbf{null}, lub inne macierzy. Dla wygody skorzystamy z pojęć \textbf{whitespace} oraz \textbf{expression} (opisane w rozdziale \emph{Wyrażenia i operatory}).

\begin{rail}
  whitespace : (([space]'' | [linefeed]'' | [horizontal tab]'')+)?;
  
\end{rail}

\hspace*{-3cm}\scalebox{0.9}{
  \vbox{
    \begin{rail}

      matrix : '[' (((whitespace+)?) | ((whitespace+)? expression (whitespace+)? + (',' | ';'))) ']';
  
    \end{rail}
  }
}

\bigskip
Znak \textbf{,} jest używany dla separacji kolejnych elementów w jednym wierszu macierzy. Znak \textbf{;} jest używany dla separacji kolejnych wierszów. Wiersz macierzy nie może być pusty, chyba, że cała macierz jest pusta. Liczba elementów w każdym wierszu macierzy musi być taka sama.

\bigskip
Przykłady:

\begin{itemize}

\item {[ ]}

\item {[1, 2, 3, 4, 5]}

\item {[1, 2, 3; 4, 5, 6; 7, 8, 9]}

\item {[\\
        1, 2;\\
        3, 4;\\
        5, 6\\
       ]}
       
\item {["String", identifier; 1.0e-0, [ ] ]}

\end{itemize}

\subsection{Wyrażenia i operatory}

\subsubsection{Operatory}
Język FML ma operatory \textbf{arytmetyczne} oraz \textbf{logiczne}.

\medskip
\begin{itemize}
  \item Arytmetyczne: 
  \begin{itemize}

    \item jednoargumentowe: 
    \textbf{-}, \textbf{!}, \textbf{!!},
  
    \item dwuargumentowe:
    \textbf{+}, \textbf{-}, \textbf{/}, \textbf{*}, \textbf{**}, \textbf{div}, \textbf{mod}.
    
  \end{itemize}
  
  \item Logiczne
  \begin{itemize}
  
    \item jednoargumentowe:
    \textbf{not}
    \item dwuargumentowe:
    \textbf{==}, \textbf{$<$}, \textbf{$<=$}, \textbf{$>$}, \textbf{$>=$}, \textbf{and}, \textbf{or} 
    \item trójargumentowy:
    \textbf{? :}
  \end{itemize}
\end{itemize}

Należy zwrócić uwagę na to, że operatory unarne \textbf{-} i \textbf{not} mają łączność lewostronną, zaś operatory \textbf{!} i \textbf{!!} -- prawostronną.

\subsubsection{Wyrażenia}

Wyrażenie może być pojedynczą zmienną, lub też pewną operacją na zmiennych. Dla ustalenia kolejności operacji mogą być stosowane nawiasy okrągłe: \textbf{(} i \textbf{)}. Wyrażenie nie może być puste, nie może zawierać samych nawiasów.

\begin{rail}

expression : (identifier | value) | ('left unary operator' expression ) | (expression 'right unary operator') | (expression 'binary operator' expression) | (expression '?' expression ':' expression) | ('('?(expression)')'? )

\end{rail}

\subsubsection{Przypisanie wartości identyfikatorowi}

Przypisanie jest dokonywane za pomocą operatora przypisania \textbf{=}. 

\begin{rail}

assignment : identifier '=' expression

\end{rail}

\subsection{Pętle, funkcje, konstrukcje warunkowe}

\subsubsection{Instrukcja}

Podstawowa operacja w pętlach, funkcjach itd.
Każda instrukcja musi być zakończona znakiem średnikiem. Na diagramie poniżej pojawia się pojęcie funcji, są opisane w podrozdziale poniżej.

\begin{rail}
      instruction : (assignment | expression | 'function call') ';';
      statement : instruction | ( '<' (instruction +)  '>' );

\end{rail}

\footnote{Uwaga: zamiast \textbf{$< >$} powinne stać znaki \textbf{$\{ \}$}. Latexu z niewiadomego mi powodu nie podoba się użycie klamrowych nawiasów.}


\end{document}